\documentclass[11pt,english,compress]{beamer}

\usepackage[utf8]{inputenc}
\usepackage{verbatim}
\usepackage{eurosym}
\usepackage{stmaryrd}

\usepackage[compatibility=false]{caption}
\usepackage{subcaption}
\usepackage{pgfplots}

\useoutertheme[subsection=false]{smoothbars}
\useinnertheme[shadow=true]{rounded}

\usecolortheme{orchid}
\usecolortheme{whale}

\setcounter{tocdepth}{2}
\setcounter{secnumdepth}{0}

\setbeamertemplate{footline}[frame number]

\title{Security in Wayland-based desktop environments}
\subtitle{Privileged clients, authorization, authentication and sandboxing!}
\author{Steve Dodier-Lazaro \& Martin Peres}
\institute{PhD student at UCL \& LaBRI}

\AtBeginSection[]{
  \begin{frame}{Summary}
  \small \tableofcontents[currentsection, hideothersubsections]
  \end{frame} 
}

\begin{document}

\setbeamertemplate{navigation symbols}{}
\setbeamertemplate{footline}[frame number]

\begin{frame}[plain,noframenumbering]
	\titlepage
\end{frame}

\section{Introduction}
\subsection{General overview}
\begin{frame}
	\frametitle{General overview of the Linux Graphics stack security}

	\begin{block}{The graphics stack before 2005}
		\begin{itemize}
			\item The X-Server provided everything:
			\begin{itemize}
				\item Modesetting (CRTC \& plane management);
				\item 2D/3D acceleration;
				\item Video rendering acceleration;
				\item Input management.
			\end{itemize}
			\item The X-Server talked to the GPU directly, as root.
		\end{itemize}
	\end{block}
\end{frame}

% \begin{frame}
% 	\begin{figure}[h]
% 		\centering
% 		\includegraphics[width=1.02\linewidth]{imgs/Linux_Graphics_Stack_2013.pdf}
% 		\caption{General overview of the Linux graphics stack}
% 	\end{figure}
% \end{frame}

\end{document}
